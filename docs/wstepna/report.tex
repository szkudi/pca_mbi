\documentclass[a4paper,12pt]{article}


\usepackage[MeX]{polski}
\usepackage[utf8]{inputenc}	% kodowanie znaków
\usepackage{indentfirst}

\usepackage{upquote}  	% zamiana cudzysłowów klasycznych (,, ") na górne (" ") w kodach źródłowych
\usepackage{graphicx}	% wstawianie obrazków
\usepackage{amsmath}
\usepackage{listings}


\def\PICSDIR{PICS}

% Ustawienie marginesów
\oddsidemargin=0.5cm
\evensidemargin=-0.5cm
\topmargin=0cm
\textwidth=16cm
\textheight=23cm

% Ładniejsze tabelki
\usepackage{booktabs}


\frenchspacing

\clubpenalty=10000		% to kara za sierotki
\widowpenalty=10000		% nie pozostawia wdów
\brokenpenalty=10000 	% nie dzieli wyrazów pomiędzy stronami


\sloppy

% Pakiety z ładnymi czcionkami
\usepackage[T1]{fontenc}
\usepackage{lmodern}


% Ładniejsze tabelki
\usepackage{booktabs}

% Ustawienia wyglądu listingów
\lstset{
	language=C++,                               % choose the default language of the code
	basicstyle=\footnotesize\ttfamily,       	% the size of the fonts that are used for the code
	numbers=left,                   		% where to put the line-numbers
%	numberstyle=\tiny,      			% the size of the fonts that are used for the line-numbers
	stepnumber=1,                   		% the step between two line-numbers. If it's 1 each line will be numbered
	numbersep=5pt,                  		% how far the line-numbers are from the code
	showspaces=false,               		% show spaces adding particular underscores
	showstringspaces=false,         	% underline spaces within strings
	showtabs=false,                 		% show tabs within strings adding particular underscores
	tabsize=4,	                		% sets default tabsize
	captionpos=b,                   		% sets the caption-position
	breaklines=true,                		% sets automatic line breaking
	breakatwhitespace=true,        	% sets if automatic breaks should only happen at whitespace
	extendedchars=true,
	keywordstyle=\bfseries,
	identifierstyle=,
	commentstyle=\slshape,
	stringstyle=\slshape,
	xleftmargin=30pt,
	frame=tb,
	framexleftmargin=30pt,
}

\begin{document}

\title{{\small Metody Bioinformatyki}\\Wykorzystanie PCA do analizy danych z mikromacierzy DNA}
\author{Maciej Czerniak, Marcin Kamionowski, Kacper Szkudlarek}

\maketitle

\begin{abstract}
Dokumentacja realizacji projektu z przedmiotu "Metody Bioinformatyki". W ramach projektu wykonane zostanie oprogramowanie wykorzystujące metodę Analizy Składowych Głównych (PCA) do przetwarzania danych uzyskanych z mikromacierzy DNA.
\end{abstract}


\section{Wstęp}
\textbf{Analiza głównych składowych} (ang. Principal Component Analysis, PCA) jest jedną ze statystycznych metod analizy czynnikowej. Zbiór danych składający się z N obserwacji, z których każda obejmuje K zmiennych, można interpretować jako chmurę N punktów w przestrzeni K-wymiarowej. Celem PCA jest taki obrót układu współrzędnych, aby maksymalizować w pierwszej kolejności wariancję pierwszej współrzędnej, następnie wariancję drugiej współrzędnej, itd.. Tak przekształcone wartości współrzędnych nazywane są ładunkami wygenerowanych czynników (składowych głównych). W ten sposób konstruowana jest nowa przestrzeń obserwacji, w której najwięcej zmienności wyjaśniają początkowe czynniki. PCA jest często używana do zmniejszania rozmiaru zbioru danych statystycznych, poprzez odrzucenie ostatnich czynników.

\textbf{Mikromacierz DNA} jest to płytka szklana lub plastikowa z naniesionymi w regularnych pozycjach mikroskopowej wielkości polami (ang. spots), zawierającymi różniące się od siebie sekwencją fragmenty DNA. Fragmenty te są sondami, które wykrywają przez hybrydyzację komplementarne do siebie cząsteczki DNA lub RNA.

\begin{figure}
\centering
\includegraphics[width=5cm]{\PICSDIR/cDNA.jpg}
\caption{Pokolorowana próbka danych z mikromacierzy cDNA}
\label{rys:cDNA}
\end{figure}

Dane (Rys: \ref{rys:cDNA}) uzyskiwane w eksperymentach prowadzonych z wykorzystaniem mikromacierzy to wartości intensywności czerwonej oraz zielonej fluorescencji każdego z pół na płytce. Jedonorazowo w eksperymencie możliwe jest badanie ekspresji kilku tysięcy genów, dlatego uzyskane dane są wysoce złożone i wielowymiarowe.

%\section{Zakładane funkcjonalności}
%
%\begin{enumerate}
%\item Wczytywanie danych z mikromacierzy.
%\item Przeprowadzenie analizy składowych głównych dla pobranych danych.
%\item Prezentacja wyników przetworzonych danych.
%\end{enumerate}

\section{Implementacja}

Implementacje podzielona będzie na kilka części:

\begin{enumerate}
\item Implementacja modułu pozwalającego na wczytywanie danych dostarczonych w surowej postaci plików tekstowych (tabbed separate data). Plik będzie analizowany pod kątem spójności i zgodności z formatem danych. Dane będą zapisywane w pamięci sposób pozwalający na ich dalszą obróbkę i analizę.
\item Stworzenie graficznego interfejsu użytkownika. Prosty interfejs pozwalający na wczytywanie danych, uruchomienie/zatrzymanie analizy, a także przejście do wizualizacji danych przed i po przetworzeniu. W implementacji zostanie wykorzystana mulitplatformowa biblioteka Qt.
\item Moduł analizatory i przetwarzający dane wejściowe. Analiza danych będzie polegała na poddaniu ich analizie składowych głównych (PCA). Wykorzystana zostanie implementacja PCA znajdująca się w bibliotece OpenCV.
\end{enumerate}

Całość implementacji zostanie wykonana w języku C/C++ pod kontrolą systemu Linux. Dzięki użyciu multiplatformowych bibliotek aplikacja będzie w pełni przenośna, wymagać będzie jedynie rekompilacji pod kontrolą systemu docelowego.
%
%\begin{itemize}
%\item Implementacja zostanie wykonana w języku C/C++ dla systemu Linux.
%\item Wykorzystanie biblioteki Qt do stworzenia GUI.
%\item Użycie gotowej implementacji PCA z pakietu OpenCV.
%\end{itemize}


\end{document}
